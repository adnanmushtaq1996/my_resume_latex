%%%%%%%%%%%%%%%%%
% This is an example CV created using altacv.cls (v1.1.5, 1 December 2018) written by
% LianTze Lim (liantze@gmail.com), based on the
% Cv created by BusinessInsider at http://www.businessinsider.my/a-sample-resume-for-marissa-mayer-2016-7/?r=US&IR=T
%
%% It may be distributed and/or modified under the
%% conditions of the LaTeX Project Public License, either version 1.3
%% of this license or (at your option) any later version.
%% The latest version of this license is in
%%    http://www.latex-project.org/lppl.txt
%% and version 1.3 or later is part of all distributions of LaTeX
%% version 2003/12/01 or later.
%%%%%%%%%%%%%%%%

%% If you are using \orcid or academicons
%% icons, make sure you have the academicons
%% option here, and compile with XeLaTeX
%% or LuaLaTeX.
% \documentclass[10pt,a4paper,academicons]{altacv}

%% Use the "normalphoto" option if you want a normal photo instead of cropped to a circle
% \documentclass[10pt,a4paper,normalphoto]{altacv}

\documentclass[10pt,a4paper,ragged2e]{altacv}

%% AltaCV uses the fontawesome and academicon fonts
%% and packages.
%% See texdoc.net/pkg/fontawecome and http://texdoc.net/pkg/academicons for full list of symbols. You MUST compile with XeLaTeX or LuaLaTeX if you want to use academicons.

% Change the page layout if you need to
\geometry{left=1cm,right=9cm,marginparwidth=6.8cm,marginparsep=1.2cm,top=1.25cm,bottom=1.25cm}

% Change the font if you want to, depending on whether
% you're using pdflatex or xelatex/lualatex
\ifxetexorluatex
  % If using xelatex or lualatex:
  \setmainfont{Carlito}
\else
  % If using pdflatex:
  \usepackage[utf8]{inputenc}
  \usepackage[T1]{fontenc}
  \usepackage[default]{lato}
\fi

% Change the colours if you want to
\definecolor{LigthBlue}{HTML}{3E0097}
\definecolor{SlateGrey}{HTML}{2E2E2E}
\definecolor{LightGrey}{HTML}{666666}
\colorlet{heading}{VividPurple}
\colorlet{accent}{VividPurple}
\colorlet{emphasis}{SlateGrey}
\colorlet{body}{LightGrey}

% Change the bullets for itemize and rating marker
% for \cvskill if you want to
\renewcommand{\itemmarker}{{\small\textbullet}}
\renewcommand{\ratingmarker}{\faCircle}

%% sample.bib contains your publications
\addbibresource{sample.bib}

\begin{document}
\name{Adnan Mushtaq Ali Karol}
\tagline{Aspriring Data Scientist / Machine Learning Engineer}
% Cropped to square from https://en.wikipedia.org/wiki/Marissa_Mayer#/media/File:Marissa_Mayer_May_2014_(cropped).jpg, CC-BY 2.0
\photo{3.0cm}{Pic_Adnan.jpg}
\personalinfo{%
  % Not all of these are required!
  % You can add your own with \printinfo{symbol}{detail}
  \email{adnanmushtaq5@gmail.com}
   \phone{+4915171684018}
  \mailaddress{Allmandring 22C , Zimmer 8}
  \location{Stuttgart-70569}
   \linkedin{Adnan Karol}
\github{adnanmushtaq1996}
  %\faXing \quad{Adnan Mushtaq Ali Karol}
  \faGlobe{  http://adnan-karol.mystrikingly.com/}
  
%   \github{github.com/mmayer} % I'm just making this up though.
%   \orcid{orcid.org/0000-0000-0000-0000} % Obviously making this up too. If you want to use this field (and also other academicons symbols), add "academicons" option to \documentclass{altacv}
}

%% Make the header extend all the way to the right, if you want.
\begin{fullwidth}
\makecvheader
\end{fullwidth}

%% Depending on your tastes, you may want to make fonts of itemize environments slightly smaller
\AtBeginEnvironment{itemize}{\small}

%% Provide the file name containing the sidebar contents as an optional parameter to \cvsection.
%% You can always just use \marginpar{...} if you do
%% not need to align the top of the contents to any
%% \cvsection title in the "main" bar.


\cvsection[page1sidebar]{Experience}

\cvevent{Master Thesis Student}{Fruanhofer IAO / ISS Universität Stuttgart}{October 2020 -- Ongoing}{Stuttgart, Germany }
\begin{itemize}
\item Handcrafted machine learning versus Deep Learning: Decoding neurophysiological signals for the comparison of workload and emotion classification.

\item Institute : In collaboration with Institut für Signalverarbeitung und Systemtheorie (ISS) at the Universität Stuttgart and Neurolab at the Fraunhofer IAO.
\item Use of proper machine Learning algorithms for feature extraction of EEG signals and comparison with Deep Learning architectures.
\item Skills : Deep Learning, Machine Learning, Feature Extraction, Signal Processing, EEG, ECG, EDA signals.

\divider

\end{itemize}

\cvevent{Werkstudent in Data Science}{Wearable Technologies AG}{April 2021 -- Ongoing}{Stuttgart/Munich, Germany }
\begin{itemize}
\item Working in field of Machine Learning and Data Science.
\item Vision is to proactively create an intelligent Wearables \& IoT landscape for the future.

\divider

\end{itemize}

\cvevent{Research Assistant}{Institute for Technical Optics, Universität Stuttgart}{March 2021 -- Ongoing}{Stuttgart, Germany }
\begin{itemize}
\item Working as Research Assistant for development of object detection algorithms for Virtual Reality glasses.
\item Use of modern architectures like YOLOv4, etc to detect and localize movement of people in a  VR based game-room.

\divider

\end{itemize}

\cvevent{Werkstudent in Data Science}{Vialytics GmbH}{July 2020 -- March 2021}{Stuttgart, Germany }
\begin{itemize}
\item Working in field of Machine Learning and Data Science.
\item Current Tasks : Train an Object Detector using YOLO v4 to detect road damages using Images and Oracle cloud.
\item Use of non-visual smartphone sensors (Acceleration) data on the Smartphone to assess the unevenness of the road (Deployed and Testing).
\item Fault and Anomaly analysis for detecting and correcting outliers categorized Roads. Deployed using gRPC and Docker.

\divider

\end{itemize}




\cvevent{Internship in Software Development and Integration}{Robert Bosch GmbH}{May 2020 -- Oct 2020}{Stuttgart, Germany }
\begin{itemize}
\item Internship in software and tool development for semi-autonomous parking systems.
\item Working with the Software Integration Team in an agile approach.
\item Skills: Python, Jenkins, QAC, and Continuous Integration, JIRA, Git.



\end{itemize}



\cvsection[page2sidebar]{Experience}
\cvevent{Werkstudent in Data Analytics}{Argonics GmbH}{April 2020 -- May 2020}{Stuttgart, Germany }
\begin{itemize}
\item Prediction and Analysis for inland navigation vessels.
\end{itemize}



\divider


\cvevent{Werkstudent in Sofware Development}{Robert Bosch GmbH}{December 2019 -- April 2020}{Stuttgart, Germany }
\begin{itemize}
\item Backend Development, Data management, data visualisation and data analysis.
\item Involved in HIL (hardware in Loop) Analysis.
\item Worked on Sequence Quality Centre tool, Mf4 and MDF Data, Python Automation Tasks.
\item Tools: Python scripting, Python GUI programming, GIT and SQL.



\end{itemize}

\divider

\cvevent{Student Research Assistant in Software Development IoT}{Fraunhofer Institute for Production Engineering and Automation IPA}{June 2019 -- Sept 2019}{Stuttgart, Germany }
\begin{itemize}
\item Implementation of wireless communication and cloud computing for exoskeletons in the convertible factory.
\item Creation of PWA of the Web Application for displaying the data from the Exoskeleton.
\item Creation of Dashboard for data visualization of the data from the Exoskeleton sensor coming as data stream using Kafka.


\end{itemize}

\divider





\cvsection{Projects}

\begin{itemize}
    \item LSTM and CNN based Human activity Classification. Keywords-LSTM, CNN, Sensor Fusion, Kalman Filter, TensorFlow, IMU, Raspberry pi.
    \item EEG Signal Classification using Hand-crafted Machine Learning and Deep Learning. Keywords-EEG, Random-Forest, CNN, Dense Networks.
        \item Machine Learning Model Deployment using Flask. Keywords-Flask, ML-Deployment.
    \item CNN and Image Procesing Based Plant Disease Detection and Remedy Using Drone.Keywords- CNN, Keras,Tensor Flow,Drone , Image Processing, Google Colab.
    \item Real-Time Object Detection using YOLOv3.
    \item Smart Desktop Voice Assistant.Implemented a Jarvis like Smart Desktop Voice Assistant.Keywords-Python Programming.
    \item Volume Unit Meter.Keywords- Arduino, Sound senor, C programming
    \item Face Detecion Using OpenCV and HaarCascades.Keywords:Computer Vision
    \item MNIST Digit Classfication Using Sequential Model.Keywords-Sequential Neural Network.
    
    
    
\end{itemize}



\end{document}